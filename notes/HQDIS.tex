\documentclass[letterpaper,11pt]{article}
\pdfoutput=1

\usepackage{jheppub}
\usepackage{multirow}
\usepackage{subfig}
\usepackage{xspace}
\usepackage{xcolor}
\usepackage[countmax]{subfloat}
\usepackage{amsmath}
\usepackage{mathtools}
\usepackage{graphicx}
\usepackage{slashed}
%\usepackage[colorlinks=true, allcolors=blue]{hyperref}
\usepackage{tikz-feynman,contour}
\usepackage{tikz,pgf}
\usepackage{braket}
\usepackage{amsfonts}
\usepackage{amsthm,scrextend,bbold}
\usepackage{comment}
%\usepackage{showkeys}

%s\usepackage{bigints}


\newcommand{\as}{\alpha_\text{s}}
\newcommand{\cf}{C_{\text{F}}}
\newcommand{\ca}{C_{\text{A}}}
\newcommand{\nc}{N_{\text{C}}}
\newcommand{\msb}{\overline{\text{MS}}}
\newcommand{\order}[1]{{\cal O}\left(#1\right)}
\newcommand{\Ob}{{\cal O}}
\newcommand{\ecf}{e_2^{(2)}}
\newcommand{\pull}{\underline{t}}
\DeclareMathOperator{\De}{d}
\newcommand{\de}{\De\!}
\newcommand{\xf}{x}
\newcommand{\tf}{t_{\text{F}}}
%\newcommand{\xt}{x_{\text{T}}}
\newcommand{\xt}{\tau}
\newcommand{\Tt}{t_{\text{T}}}
\newcommand{\xb}{x_{\text{B}}}
\newcommand{\chib}{\chi_{\text{B}}}
\newcommand{\tb}{t_{\text{B}}}
\newcommand{\gs}{\gamma_\text{soft}}
\newcommand{\gszero}{\gamma_\text{soft}^{(0)}}
\newcommand{\gszeroMm}{\gamma_\text{soft1}^{(0)}}
\newcommand{\gszerotilde}{\widetilde{\gamma}_\text{soft}^{(0,\text{sub})}}
\newcommand{\gszerotildeMm}{\gamma_\text{soft1}^{(0,\text{sub})}}
\newcommand{\gsone}{\gamma_\text{soft}^{(1)}}
\newcommand{\cone}{C^{(1)}}
\newcommand{\fonll}{FO(NLL)$^2$ }
\newcommand{\muf}{\mu_{\text{F}}}
\newcommand{\muOf}{\mu_{0\text{F}}}
\newcommand{\mur}{\mu_{\text{R}}}
\newcommand{\muOr}{\mu_{0\text{R}}}
\newcommand{\bzerofour}{\beta_0^{(4)}}
\newcommand{\bzerofive}{\beta_0^{(5)}}
\newcommand{\bonefour}{\beta_1^{(4)}}
\newcommand{\bonefive}{\beta_1^{(5)}}
\usepackage{color}
\definecolor{darkblue}{rgb}{0,0,0.5}
\definecolor{darkgreen}{rgb}{0,0.5,0}
\definecolor{darkorange}{rgb}{0.8,0.3,0}
\newcommand{\ag}[1]{\textbf{\textcolor{darkorange}{(#1 --ag)}}}
\newcommand{\sm}[1]{\textbf{\textcolor{darkgreen}{(#1 -sm)}}}
\newcommand{\gr}[1]{\textbf{\textcolor{cyan}{(#1 -gr)}}}



\DeclareRobustCommand{\Sec}[1]{Sect.~\ref{#1}}
\DeclareRobustCommand{\Secs}[2]{Secs.~\ref{#1} and \ref{#2}}
\DeclareRobustCommand{\App}[1]{App.~\ref{#1}}
\DeclareRobustCommand{\Tab}[1]{Table~\ref{#1}}
\DeclareRobustCommand{\Tabs}[2]{Tables~\ref{#1} and \ref{#2}}
\DeclareRobustCommand{\Fig}[1]{Fig.~\ref{#1}}
\DeclareRobustCommand{\Figs}[2]{Figs.~\ref{#1} and \ref{#2}}
\DeclareRobustCommand{\Eq}[1]{Eq.~(\ref{#1})}
\DeclareRobustCommand{\Eqs}[2]{Eqs.~(\ref{#1}) and (\ref{#2})}
\DeclareRobustCommand{\Ref}[1]{Ref.~\cite{#1}}
\DeclareRobustCommand{\Refs}[1]{Refs.~\cite{#1}}

\newcommand{\pythia}[1]{\textsc{Pythia\xspace #1}}
\newcommand{\madgraph}[1]{\textsc{MadGraph5\xspace #1}}
\newcommand{\fastjet}[1]{\textsc{FastJet\xspace #1}}
\newcommand{\herwig}[1]{\textsc{Herwig\xspace #1}}
\newcommand{\herwigpp}[1]{\textsc{Herwig++\xspace #1}}
\newcommand{\vincia}[1]{\textsc{Vincia\xspace #1}}

\newcommand{\mathematica}{\texttt{Mathematica }}
\newcommand{\hypexp}{\texttt{HypExp }}
\newcommand{\eventtwo}{\texttt{EVENT2}}

\bibliographystyle{JHEP}

%\preprint{}


\title{
CC DIS
}

\author[1]{Andrea Ghira,}
\author[1]{Simone Marzani,}
\author[1]{Giovanni Ridolfi,}
\author[2]{and Maria Ubiali}

\affiliation[1]{Dipartimento di Fisica, Universit\`a di Genova and INFN, Sezione di Genova,Via Dodecaneso 33, 16146, Italy}
\affiliation[2]{DAMTP, University of Cambridge, Wilberforce Road, Cambridge, CB3 0WA, United Kingdom}
\emailAdd{andrea.ghira@ge.infn.it}
\emailAdd{simone.marzani@ge.infn.it}
\emailAdd{giovanni.ridolfi@ge.infn.it}
\emailAdd{mu227@cam.ac.uk}

\abstract{}

\begin{document}
\maketitle


\section{Introduction}\label{sec:intro}


%%%%%%%%%%%%%%%%%%%%%%%%%%%%%%%%%%%%%%%%%%%%%
\section{Extension to other processes}\label{sec:extensions}

Thus far, we have focussed our discussion on the decay of a of colour singlet into a heavy-quark pairs. However, our calculation relies upon factorisation properties of QCD matrix elements in the quasi-collinear and soft limits, which are universal. Therefore, our formalism can be applied to different processes. In this section, we discuss some applications, highlighting the structure of parton-level results. This will enable us to shed light on issues pointed out in the literature. In this context, we will also discuss similarities and differences with other approaches. 
%
Instead, we will leave phenomenological (hadron-level) studies to future work.

\subsection{Deep inelastic scattering with heavy flavours}\label{sec:dis}

The first process we would like to discuss is deep-inelastic scattering (DIS) with heavy flavours. In particular, we focus on charged current (CC) DIS. Our discussion applies to to the scattering of a neutrino $\nu$ off a nucleus $N$, $\nu N \to l X$, where $l$ is a charged lepton, as well as to CC  $l N \to \nu X$ scattering. In both cases, we are interested in the situation in which charm flavour is identified in the final state. The parton-model contribution to these processes is given by
\begin{equation}
	q(p_1)+W^*(q)\to c (p_2)\quad p_1^2=0,\quad p_2^2=m^2,
\end{equation}
where the initial-state quark is taken massless.~\footnote{Neutral-current DIS with heavy flavours is also interesting, especially in the context  of studies about the intrinsic heavy-flavour component of the proton wavefunction.} Note that in this section we use $m$ to indicate the charm quark mass and we define the related adimensional ratio $\xi=\frac{m^2}{Q^2}$, with $Q^2=-q^2>0$.


%
It is interesting to note that CC-DIS experiments cover a rather large kinematical range in both the Bjorken scaling variable $x_\text{B}= \frac{Q^2}{2 P\cdot q}$ (where $P$ is the momentum of the nucleon) and the momentum transferred $Q^2$. 
Depending on the relative size of the charm quark mass $m^2$ and $Q^2$, a massless calculation or a massive one maybe more appropriate. 
%
Furthermore, if we want to investigate the large $x_{\text{B}}$ as done, for instance, in fixed-target experiments, then we face the issue of consistently combining the resummation of collinear logarithms and large-$x$ logarithms.
%
This problem was first investigated in Ref.~\cite{Corcella:2003ib}, where two separate formulations of soft-gluon resummation were derived: one valid in massless limit $m^2 \ll Q^2$ and one that kept the full charm-mass dependence, appropriate for the region $m^2 \sim Q^2$. 
%
The Authors of Ref.~\cite{Corcella:2003ib} noted that the non-commutativity of the $m^2/Q^2 \to 0$ and $N \to \infty$ limits prevented them from obtaining a single resummed formula. 
%
%
Thanks to the formalism we have developed, we are now able to bridge the gap between these two formulations, arriving at a unified resummed expression that interpolates between the two regimes.

We start by reviewing the basic framework used to describe CC DIS. We try to follow as much as possible the notation adopted in~\cite{Corcella:2003ib}. For this reason, we consider neutrino scattering (a similar expression holds for lepton scattering) and write the double differential cross section in terms of three structure functions $F_1, F_2$ and $F_3$:
\begin{align}\label{eq:sigma DIS CC}
		\frac{\de^2\sigma^{\nu (\bar{\nu})}}{\de \xb \de y}= \frac{G_{\text{F}}M E}{\pi \left(1+\frac{Q^2}{M_W^2}\right)^2}\left\{y^2 \xb F_1+\left[1-\left(1+\frac{M \xb}{2 E}\right)\right]F_2\pm y\left(1-\frac{y}{2}\right)\xb F_3\right\}.
\end{align}
Here $G_F$ is the Fermi constant, $M$ the mass of the nucleon, $E$ is the neutrino energy in the nucleon rest frame, and $y= \frac{P\cdot q}{P\cdot l_1}$,
with $l_1$ the momentum of the incoming lepton. The parton model and collinear factorisation allow us to express the structure functions $F_i$ in terms of perturbative coefficients functions and universal parton distribution functions (PDFs). However, in the presence of heavy quarks, the parton model needs to be generalised~\cite{Aivazis:1993kh,Aivazis:1993pi,Kretzer:1998ju,Collins:1998rz}
%
In particular, one usually introduces new scaling variables,
\begin{equation}
	\chi_{\text{B}}=\frac{\eta}{Q^2}\left(Q^2+m^2\right), \quad 	\frac{1}{\eta} = \frac{1}{2\xb}+\sqrt{\frac{1}{4\xb^2}+\frac{M^2}{Q^2}},
\end{equation}
which both reduce to the standard $x_\text{B}$ scaling variable when $m, M \to 0$.
%
If we parametrise the momentum of the nucleon in light-cone coordinates as $P^\mu=(P^+,\vec{0},\frac{M^2}{2 P^+})$, then, in the generalised parton model, the momentum of the incoming parton is assumed to be $p_1^\mu=(\zeta P^+,\vec{0},0)$. 
%
At tree-level, this leads to $\chi=\zeta$, so $\chi$ can be interpreted as the ``energy'' fraction of the incoming quark respect to the nucleon \cite{Aivazis:1993kh}.\footnote{The energy fraction is usually labelled $\xi$ in the literature, but we here we use $\chi$ to avoid confusion with $\xi$ previously defined.}
%
The analysis of higher-order corrections leads to a generalisation of collinear factorisation, in the presence of heavy quarks:
\begin{equation} \label{eq:coll-fact-HQ}
	\mathcal{F}^i(\chib,\xi)=\int^1_{\chib} \frac{\de \zeta}{\zeta} \left[C^i_q\left(\frac{\chib}{\zeta},\muf^2,\xi\right) f_q(\zeta,\muf^2)+C^i_g\left(\frac{\chib}{\zeta},\muf^2,\xi\right) f_g(\zeta,\muf^2)\right],
\end{equation}
for $i=1,2,3$. 
The $\mathcal{F}^i$ functions are related to the structure functions $F_i$ through:
\begin{equation}
		\mathcal{F}_1=F_1, \quad 
		\mathcal{F}_2=\frac{2 \xb(m^2+Q^2)}{Q^2+4M^2 \xb^2} F_2, \quad 
		\mathcal{F}_3=\frac{2}{\sqrt{1+\frac{4 M^2 \xb^2}{Q^2}}} F_3.
\end{equation}
In Eq.~(\ref{eq:coll-fact-HQ}), $C_i^q$ and $C_i^g$ are the coefficient functions the quark- and gluon-initiated processes, and $f_q$ and $f_g$ are the corresponding PDFs.
%
Note that the coefficient functions depends on the charm mass through $\xi$ and that their scaling variable is $x=\frac{\chib}{\zeta}$.

The quark-initiated contribution is enhanced in the soft limit and NLL resummation was performed in~\cite{Corcella:2003ib} by considering Mellin moments of the coefficient functions $C_i^q$:
\begin{align}\label{eq:resummed_DIS_massive}
\widetilde{C}^i_q\left(N,\frac{Q^2}{\muf^2},\xi\right)&= \int_0^1 \de  x \, x^{N-1}
C^i_q\left(x,\muf^2,\xi\right) 
=\left(1+ \frac{\as(\mur^2)\cf }{\pi}\mathcal{H}^i_1\left(\frac{Q^2}{\muf^2},\xi\right)\right) \nonumber \\
&
\exp\left\{-\int^{1}_{\frac{1}{\bar N}} \frac{\de z}{z} \left[\int^{\mathcal{M}^2 z^2}_{\muf^2} \frac{\de k_t^2}{k_t^2} A\left(\as(k_t^2)\right)+H\left(\as\left(\mathcal{M}^2z^2\right)\right)\right]\right\},
\end{align}
where, for NLL accuracy, we need $A_1,A_2,H_1$, see Eq.~(\ref{Resummation_constants}), and
\begin{align}
		\mathcal{H}^i_1\left(\frac{Q^2}{\muf^2},\xi\right)=& \frac{3}{4} \log{\left(\frac{Q^2(1+\xi)}{\muf^2}\right)}+\log{\left(\frac{\xi}{1+\xi}\right)}\frac{1-2\xi}{4}\\
		&+\frac{h_i}{2}-\text{Li}_2\left(-\frac{1}{\xi}\right)-2.
\end{align}
The constant $\textit{h}_i$ are~\cite{Gluck:1996ve}:
\begin{equation}
h^1=h^3=0,\quad h^2= \xi \log{\left(\frac{\xi}{1+\xi}\right)}.
\end{equation}
The scale that appears in the upper limit of the transverse momentum integration is 
\begin{equation}\label{eq:DIS_mass_scale}
\mathcal{M}^2=Q^2\frac{(1+\xi)^2}{\xi}.
\end{equation}
%
The integrals in Eq.~(\ref{eq:resummed_DIS_massive}) can be evaluated to NLL accuracy. Explicit results can be found in Ref.~\cite{Corcella:2003ib}. The structure of the resummation  Eq.~(\ref{eq:resummed_DIS_massive}) presents some differences with respect to the one for $H\to b\bar b$ previously described and, therefore, it deserves some comments. 
%
Even if it is performed by keeping the final-state charm quark massive, because the initial-state quark is massless, the resummed exponent features both double logarithmic contributions, related to gluon emissions off massless partons, and contributions that the product of logarithms of $N$ and $\xi$, originating from emission off the massive charm.
%
Furthermore, we note that in the small mass limit, the coefficient $\mathcal{H}^i_1$ has a double logarithmic behaviour
\begin{equation}
\mathcal{H}^i_1=\frac{1}{2}\log^2{\xi} +\frac{3}{4} \log \frac{Q^2}{\muf^2}-\frac{1}{4} \log \xi -\frac{\pi^2}{6}
% \text{Li}_2\left(-\frac{1}{\xi}\right)=-\frac{1}{2}\log^2{\xi}+\frac{\pi^2}{6}
 +\order{\xi}.
\end{equation}
As we have already pointed out, the mass of the unresolved particle (here the charm quark) is responsible of the double mass logarithm. 
%
%We have explicitly checked this by considering $\mathcal{O}(\as)$ coefficient functions with different (non-zero) masses~\cite{Kretzer:1998ju} and performing the massless limit.


In Ref.~\cite{Corcella:2003ib} the resummed calculation with massless charm, was also performed. The result reads
\begin{align}\label{eq:resummed_DIS_massless}
\widetilde{G}^{i}_q\left(N,\frac{Q^2}{\muf^2}\right)&
=\left(1+ \frac{\as(\mur^2)\cf }{\pi}\mathcal{G}^i_1\left(\frac{Q^2}{\muf^2}\right)\right) \nonumber \\
&
\exp\left\{-\int^{1}_{{\frac{1}{ \bar N}}} \frac{\de z}{z} \left[\int^{Q^2 z}_{\muf^2} \frac{\de k_t^2}{k_t^2} A\left(\as(k_t^2)\right)+\frac{1}{2} B\left(\as\left(Q^2z\right)\right)\right]\right\},
\end{align}
where for NLL accuracy, we need $A_1,A_2,B_1$, see Eq.~(\ref{Resummation_constants}), and
\begin{align}
		\mathcal{G}^i_1\left(\frac{Q^2}{\muf^2}\right)=& \frac{3}{2} \log \frac{Q^2}{\muf^2}-\frac{\pi^2}{12}-\frac{9}{4}.
\end{align}
Once again, we find that Eq.~(\ref{eq:resummed_DIS_massless}) cannot be obtain by taking the $\xi \to 0$ limit of Eq.~(\ref{eq:resummed_DIS_massive}).

In order to apply our resummation formalism, we have to write the above results in terms of jet functions. 
%
In DIS experiments, a measurement of $\chi_\text{B}$ probes the dynamics of the incoming particle.
%
Therefore, the measured jet function $J_q$ describes radiation collinear to the initial-state quark and it is therefore massless in both cases:
\begin{equation}\label{eq:massless-jet-function}
J_q\left(N,\frac{Q^2}{\muf^2}\right)=-\int^{1}_{{\frac{1}{ \bar N}}} \frac{\de z}{z} \int^{Q^2 z^2}_{\muf^2} \frac{\de k_t^2}{k_t^2} A\left(\as(k_t^2)\right).
\end{equation}
%
The final state instead is treated differently in the two cases. If we consider the charm quark as massless, as in Eq.~(\ref{eq:resummed_DIS_massless}), then the charm jet is equal to the unmeasured jet function $\bar J$, defined in Eq.~(\ref{eq: barJ}) (with $Q^2$ instead of $q^2$). Thus, we can rewrite Eq.~(\ref{eq:resummed_DIS_massless}) as
\begin{align}\label{eq:resummed_DIS_massless_bis}
\widetilde{G}^{i}_q\left(N,\frac{Q^2}{\muf^2}\right)&
=\left(1+ \frac{\as(\mur^2)\cf }{\pi}\mathcal{G}^i_1\left(\frac{Q^2}{\muf^2}\right)\right) \exp\left [ J_q\left(N,\frac{Q^2}{\muf^2}\right)+\bar J(N)\right].
\end{align}
On the other hand, in the massive case we have 
\begin{align}\label{eq:resummed_DIS_massive_bis}
\widetilde{C}^i_q\left(N,\frac{Q^2}{\muf^2},\xi\right)&= \left(1+ \frac{\as(\mur^2)\cf }{\pi}\mathcal{H}^i_1\left(\frac{Q^2}{\muf^2},\xi\right)\right) \nonumber \\
&
\exp\left\{J_q\left(N,\frac{Q^2}{\muf^2}\right)-\int^{1}_{\frac{1}{\bar N}} \frac{\de z}{z} \left[\int^{\mathcal{M}^2 z^2}_{Q^2 z^2} \frac{\de k_t^2}{k_t^2} A\left(\as(k_t^2)\right)+H\left(\as\left(\mathcal{M}^2z^2\right)\right)\right]\right\},
\nonumber\\
&= \left(1+ \frac{\as(\mur^2)\cf }{\pi}\mathcal{H}^i_1\left(\frac{Q^2}{\muf^2},\xi\right)\right) \nonumber \\
&
\exp\left\{J_q\left(N,\frac{Q^2}{\muf^2}\right)-2 \int^{1}_{\frac{1}{\bar N}} \frac{\de z}{z}  \as\left(\mathcal{M}^2z^2\right)
	\gszeroMm(\xi)\right\}+\text{NNLL},
\end{align}
where we have introduced the appropriate soft anomalous dimension for one massive and one massless parton, see e.g.~\cite{Kidonakis:2020gxo}  
\begin{equation}
 	\gszeroMm(\xi)=\frac{\cf}{2}\left[\log{\left(\frac{(1+\xi)^2}{\xi}\right)}-1\right].
 \end{equation} 

Our goal is to match Eq.~(\ref{eq:resummed_DIS_massless_bis}) to Eq.~(\ref{eq:resummed_DIS_massive_bis}).
%
First, we note that the massless jet function $J_q$ is the same in both Eq.~(\ref{eq:resummed_DIS_massless_bis}) and Eq.~(\ref{eq:resummed_DIS_massive_bis}). 
%
The evaluation of integrals in Eq.~(\ref{eq:massless-jet-function}) requires some care if we want to keep track of quark flavour thresholds, as discussed at length in the previous sections.
%
Note that, in contrast to the heavy-quark fragmentation case, where we had a perturbative initial condition, here we must supplement the description of radiation collinear to the initial state with non-perturbative PDFs, which also require specifying the flavour number scheme. Following Ref.~\cite{Corcella:2003ib}, we do not include any DGLAP evolution factor in $J_q$, in contrast to what we did for the fragmentation case. 

Thus, the only non-trivial step which is required to match Eq.~(\ref{eq:resummed_DIS_massless_bis}) to Eq.~(\ref{eq:resummed_DIS_massive_bis}) is the calculation of the the unmeasured jet function in the quasi-collinear limit. This is precisely what was done in Sect.~\ref{sec:jbar-lund}. Therefore, the construction of the resummed structure functions follows from what was done in sec~ \ref{sec:results_decay}. 
We obtain: 
%
%\begin{equation}\label{eq:final-result-dis}
%\mathcal{F}^i_q(x,\xi)= \int_{c-i \infty}^{c+ i \infty}\frac{\de N}{2 \pi i}\, \chi^{-N} f_q(N,\muf^2)
%\begin{cases} 	
%\widetilde{G}^{i}_q\left(N,\frac{Q^2}{\muf^2}\right) & \text{if}\;1-\chib>\sqrt{\xi} ,\\
%\mathcal{F}_q^{i(\text{match})}\exp{[J_q(N)+\bar{J}^{(2)}(N,\xi)]} & \text{if}\;\sqrt{\xi}>1-\chib>\xi\; , \\
%\widetilde{C}^{i,\text{sub}}_q\left(N,\frac{Q^2}{\muf^2}\right) \exp\left[\bar{J}^{(3)}(N,\xi)\right]  &\text{if} \;\xi>1-\chib ,\\
%\end{cases}	
%\end{equation}
\begin{equation}\label{eq:final-result-dis-start}
\mathcal{F}^i_q(\chi_{\text{B}},\xi)= \int_{c-i \infty}^{c+ i \infty}\frac{\de N}{2 \pi i}\, \chi_{\text{B}}^{-N} f_q(N,\muf^2)\Sigma\left(N, \frac{Q^2}{\muf^2},\xi\right),
\end{equation}
with
\begin{equation}\label{eq:final-result-dis}
\Sigma=
\begin{cases} 	
\widetilde{G}^{i,\text{sub}}_q\left(N,\frac{Q^2}{\muf^2}\right)\exp{[J^{(1)}_q(N)+\bar{J}^{(1)}(N,\xi)]}  &  \text{if}\;1-\chib>\sqrt{\xi} ,\\ 
\mathcal{F}_q^{i,\text{match}}\exp{[J^{(2)}_q(N,\xi)+\bar{J}^{(2)}(N,\xi)]} & \text{if}\;\sqrt{\xi}>1-\chib>\xi\; , \\ 
\widetilde{C}^{i,\text{sub}}_q\left(N,\frac{Q^2}{\muf^2}\right)\exp{[J^{(2)}_q(N,\xi)+\bar{J}^{(3)}(N,\xi)]} &\text{if} \;\xi>1-\chib ,\\
\end{cases}	
\end{equation}
where we have introduced the subtracted coefficient functions
\begin{align}\label{eq:dis-subtracted}
\widetilde{G}^{i,\text{sub}}_q\left(N,\frac{Q^2}{\muf^2}\right) &=\left(1+ \frac{\as(\mur^2)\cf }{\pi}\mathcal{G}^i_1\left(\frac{Q^2}{\muf^2},\xi\right)\right),
\nonumber\\
\widetilde{C}^{i,\text{sub}}_q\left(N,\frac{Q^2}{\muf^2}\right) &=
 \left(1+ \frac{\as(\mur^2)\cf }{\pi}\mathcal{H}^i_1\left(\frac{Q^2}{\muf^2},\xi\right)\right)
\exp\left[ -2\int^{1}_{\frac{1}{\bar N}} \frac{\de z}{z}  \as\left(\mathcal{M}^2z^2\right) \gszerotildeMm(\xi) \right],
\end{align}
with
\begin{equation}
\gszerotildeMm(\xi)=\cf\log{\left(1+\xi\right)}.
\end{equation}
The jet functions introduced in Eq.~(\ref{eq:final-result-dis}) are defined as follows:
the function $J^{(1)}_q$ coincides with Eq.~(\ref{eq:massless-jet-function}) and it is the same as $\Delta(N)$ introduced in Eq~(\ref{eq: delta function}):
\begin{equation}
	J^{(1)}_q(N)=\int^{1}_{{\frac{1}{ \bar N}}} \frac{\de z}{z} \int^{\muf^2}_{Q^2 z^2} \frac{\de k_t^2}{k_t^2} A\left(\as(k_t^2)\right).
\end{equation}
As we have discussed at length in the previous sections, when we cross the threshold $1-\chi_{\text{B}}=\sqrt{\xi}$, we obtain two contributions, one at $n_l$ flavours and another at $n_f$: \sm{questo risultato non mi torna}
\begin{equation}\label{eq: J2}
 		J^{(2)}_q(N,\xi)=J^{(1)}_q\left(\frac{e^{-\gamma_E}}{\sqrt{\xi}} \right)+ \int^{\sqrt{\xi}}_{\frac{1}{\bar N}} \frac{\de z}{z} \int^{m^2}_{Q^2 z^2} \frac{\de k_t^2}{k_t^2}  A\left(\as(k_t^2)\right).
 \end{equation}
% \begin{equation}\label{eq: J2}
% 	\begin{split}
% 		J^{(2)}_q(N,\xi)=J^{(1)}_q\left(\frac{e^{-\gamma_E}}{\sqrt{\xi}} \right)- \int^{\sqrt{\xi}}_{\frac{1}{\bar N}} \de z \int^{m^2}_{q^2 z^2} \frac{\de k_t^2}{k_t^2} \frac{\as^{\text{CMW}}(k_t^2)}{2\pi}P_{qq}(z),
% 		\end{split}
% \end{equation}
Explicit results are given in App.~\ref{app:calculations}, Eq.~(\ref{eq:JqDISfinal}).
 %
The unmeasured jet functions in Eq.~(\ref{eq:final-result-dis}), $\bar{J}^{(1)}(N),\bar{J}^{(2)}(N,\xi),\bar{J}^{(3)}(N,\xi)$, coincide with the ones computed for the decay process and can be found in App.~\ref{app:calculations},
% in  Eqs~(\ref{eq:Jbar_FF},\ref{eq:Jbar_intermediate_region},\ref{eq:Jbar-final-region})  respectively and evaluated for $1-x=\frac{1}{\bar N}$. 
%The explicit expressions for $\bar{J}^{(2)}(N,\xi)$ and $\bar{J}^{(3)}(N,\xi)$ can be found in App.~\ref{app:calculations}, Eqs.~(\ref{eq: Jbar2},\ref{eq: Jbar3}).
Furthermore, $\mathcal{F}_q^{i,\text{match}}$ is an arbitrary matching constant that we cannot determine within our approximations. 
We also note that if we are interested in the large $Q^2$ region, we have to consider additional flavour thresholds in the QCD running coupling, corresponding, for instance, to the $4 \to 5$ flavour transition.
%

In order to perform phenomenological studies of Eq.~(\ref{eq:final-result-dis}), we also have to consider the role of PDFs and the effects that a variable flavour number scheme, such as the one adopted here, have on their evolution. We leave such investigations to future work. 

Finally, for the DIS jet function $J_q$, we have
\begin{equation}
	\begin{split}
	\label{eq:JqDISfinal}
J_q^{(1)}\left(N,\frac{\mur^2}{q^2}\right)&= \Delta^{(1)}\left(N,\frac{\mur^2}{q^2}\right), \\
 J_q^{(2)}\left(N,\xi,\frac{\muf^2}{q^2},\frac{\mur^2}{q^2},\frac{\muOr^2}{m^2}\right)&= \Delta^{(1)}\left(\frac{e^{-\gamma_E}}{\sqrt{\xi}} \right)+\Delta^{(2)}(N,\xi)+\hat E \left(N,\xi,m^2,\muf^2\right),
\end{split}
\end{equation}
where, for DIS, $\xi=\frac{m^2}{Q^2}$.




\bibliography{HQDIS}

\end{document} 
